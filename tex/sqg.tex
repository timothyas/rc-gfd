\section{Surface Quasi-Geostrophic Turbulence}
\label{sec:sqg}

Purpose: show current state of high resolution, large-scale RNN prediction,
highlight challenges

Capabilities:
\begin{itemize}
    \item Prediction skill compared to persistence, AR1, (maybe) coarse SQG
    \item Plots showing sample prediction somewhat capturing some dynamical
        features, but not all
    \item (TBD) Quantify with fractal dimension
\end{itemize}

Challenges:
\begin{itemize}
    \item Overly smooth predictions, highlight with spectrum. This could be due
        to many scales of turbulence, see spectrum of prediction made with
        spatially averaged data.
    \item Conservation of quantities (maybe, but we don't really address this
        with these results)
\end{itemize}

Considerations for real data:
\begin{itemize}
    \item How much data do we need? Many reanalyses available for decades, and
        as always less data = more tractable
    \item What happens when we only use one variable (e.g. just one layer)? akin
        to single variable in multi-dimensional field
    \item What about using time averaged output? (rarely do we ever have
        by-the-timestep data.
\end{itemize}
