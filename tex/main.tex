\documentclass[draft]{agujournal2019}
\usepackage{url} %this package should fix any errors with URLs in refs.
\usepackage{lineno}
\usepackage[inline]{trackchanges} %for better track changes. finalnew option will compile document with changes incorporated.
\usepackage{soul}
\linenumbers

% --- My additions
\usepackage{amsmath}
\usepackage{amssymb}
\usepackage{mathtools} % for coloneqq
\usepackage{todonotes}
\usepackage{overpic}
\usepackage{rnn}
\usepackage[capitalise,noabbrev]{cleveref}

\crefname{appendix}{}{} % gets rid of Appendix Appendix A

\newcommand{\red}[1]{\textcolor{red}{#1}}
\newcommand{\blue}[1]{\textcolor{blue}{#1}}

\newcommand{\citep}{\cite}
\newcommand{\citet}{\citeA}

%%%%%%%
% As of 2018 we recommend use of the TrackChanges package to mark revisions.
% The trackchanges package adds five new LaTeX commands:
%
%  \note[editor]{The note}
%  \annote[editor]{Text to annotate}{The note}
%  \add[editor]{Text to add}
%  \remove[editor]{Text to remove}
%  \change[editor]{Text to remove}{Text to add}
%
% complete documentation is here: http://trackchanges.sourceforge.net/
%%%%%%%

\draftfalse
\journalname{Journal of Advances in Modeling Earth Systems (JAMES)}


\begin{document}

\title{Temporal Subsampling Diminishes Small Scales in
    Recurrent Neural Network Emulators of
    Geophysical Turbulence}

%% ------------------------------------------------------------------------ %%
%
%  AUTHORS AND AFFILIATIONS
%
%% ------------------------------------------------------------------------ %%

% Example: \authors{A. B. Author\affil{1}\thanks{Current address, Antartica}, B. C. Author\affil{2,3}, and D. E.
% Author\affil{3,4}\thanks{Also funded by Monsanto.}}

\authors{
Timothy A. Smith\affil{1,2},
Stephen G. Penny\affil{1,3},
Jason A. Platt\affil{4},
Tse-Chun Chen\affil{1,2}
}
\affiliation{1}{Cooperative Institute for Research in Environmental Sciences
    (CIRES) at the University of Colorado Boulder, Boulder, CO, USA
}
\affiliation{2}{Physical Sciences Laboratory (PSL), National Oceanic and
    Atmospheric Administration (NOAA), Boulder, CO, USA
}
\affiliation{3}{Sofar Ocean Technologies, San Francisco, CA, USA}
\affiliation{4}{University of California San Diego (UCSD), La Jolla, CA, USA}

\correspondingauthor{Timothy A. Smith}{tim.smith@noaa.gov}

\begin{keypoints}
    \item Reducing training data temporal resolution by subsampling leads to
        overly dissipative small spatial scales in neural network
        emulators
    \item A quadratic autoregressive architecture is shown to be inadequate at capturing
        small scale turbulence, even when data are not subsampled
    \item Echo state networks perform better, and subsampling bias is mitigated
        but not eliminated by prioritizing kinetic energy spectrum in training
\end{keypoints}

%% ------------------------------------------------------------------------ %%
%
%  ABSTRACT and PLAIN LANGUAGE SUMMARY
%
%% ------------------------------------------------------------------------ %%

\begin{abstract}
    The immense computational cost of traditional numerical weather and climate models has sparked the development of machine learning (ML) based emulators or surrogate
    models. Because ML methods benefit from long records of training data,
    it is common to use datasets that are temporally subsampled relative to the time steps required by numerical integration of differential equation models.
    Here, we investigate how this often overlooked processing step affects
    the quality of an emulator's predictions. We implement two ML architectures from a class of methods called reservoir computing: (1) a form of Nonlinear Vector Autoregression (NVAR), and (2) an Echo State Network (ESN).
    Despite their simplicity, it is well documented that these architectures
    excel at predicting low dimensional chaotic dynamics. We are therefore motivated to test these architectures in an idealized setting of predicting high dimensional geophysical turbulence as represented by Surface Quasi-Geostrophic dynamics.
    In all cases, we see that subsampling the training data consistently leads to a bias at small spatial scales that resembles numerical diffusion.
    Interestingly, the NVAR architecture becomes unstable when the temporal
    resolution is increased, indicating that the polynomial based interactions
    are insufficient at capturing the detailed nonlinearities present in the presence of turbulence.
    The ESN architecture is found to be more robust, suggesting a benefit to the more expensive but more general structure.
    We show that spectral errors can be reduced by including a penalty on the kinetic energy density spectrum during training, although the subsampling related errors persist.
    Future work is warranted to understand how the temporal resolution
    of training data affects other neural network architectures.
\end{abstract}

\section*{Plain Language Summary}

The computer models that govern weather prediction and climate projections
are extremely costly to run, causing practitioners to make unfortunate
tradeoffs between accuracy of the physics and credibility of their statistics.
Recent advances in machine learning have sparked the development of neural
network-based emulators, i.e., low-cost models that can be used as drop-in replacements for the traditional expensive models. Due to the cost of storing large weather and climate datasets, it is common to subsample these fields in time to save disk space and reduce computational expense during training. Here, we show that this pre-processing step hinders the fidelity of the emulator. We offer one method to mitigate the resulting errors, but we suggest that more research is needed to understand and eventually overcome them.

%% ------------------------------------------------------------------------ %%
%
%  TEXT
%
%% ------------------------------------------------------------------------ %%

\section{Introduction}
\label{sec:intro}

% Climate and weather models are expensive, and we really want ensembles
Weather and climate prediction requires the integration of a computational
forecast model, derived from the fundamental equations of motion and initialized
with an estimate of the present-day system state (e.g., temperature, wind speeds,
etc.).
Our knowledge of these initial conditions is imperfect, however, and the governing
equations contain necessary approximations of reality.
Thus, reliable climate projections and weather forecasts require an
ensemble of numerical model integrations, where each ensemble member is
initialized by sampling from a prior distribution
representing our uncertainty in the present system state.
This process comes at an immense computational cost.
On the one hand, it is desirable to increase the credibility of the underlying
numerical model as much as possible,
for instance by increasing model grid resolution or by explicitly
simulating as many coupled components (e.g., atmosphere, land, ocean, ice) as
possible.
On the other hand, producing an ensemble with reliable statistical significance
requires integrating the underlying numerical model many times; usually
$\mathcal{O}(10-100)$ in practice, but ideally $>1,000$.
Therefore, the resulting computational costs require practitioners to trade between the
fidelity of the numerical model and the size of the ensemble.

% Surrogate modeling an encouraging path, and with increased computing power,
% ML, NNs ...
A current line of research for enabling statistical forecasting with an
expensive numerical model is \textit{surrogate modeling}.
The general approach involves deriving a surrogate model or emulator which
can be evaluated much faster than the original numerical model, while
capturing the dynamics of the underlying system ``accurately enough'' for
reliable prediction.
Historically, engineering applications have made use of
reduced order models \citep{moore_linear_2022},
polynomial chaos expansions,
...
More recently, advances in computing power, the explosion of freely available data, and
more widespread usage of General Purpose Graphics Processing Units (GPGPUs) has
encouraged the exploration of using Machine Learning (ML) methods like neural
networks for the emulation task.
Within the broad scope of weather forecasting and climate projection
applications, many \red{research groups} have begun to develop
neural network-based surrogate models to represent the
general circulation of the atmosphere and oceans, see \cref{table:all_the_emulators}.


\begin{table}[]
\caption{
    Recent work using neural networks to emulate processes relevant to weather
    forecasting and climate projection.
}
\label{table:all_the_emulators}

{\footnotesize
\begin{tabular}{c|c|c|c|c|l}
System     & Data Source             & \begin{tabular}[c]{@{}c@{}}Process or\\ Variable\end{tabular} & \begin{tabular}[c]{@{}c@{}}Horizontal\\ Resolution\end{tabular} & Timestep & Citation                                            \\
\hline
Atmosphere & ``Idealized GCM''       &                                                               &                                                                 &          & \citep{scher_weather_2019}       \\
           & 2 Layer QG              &                                                               &                                                                 &          & \citep{chattopadhyay_deep_2020}  \\
           & SPEEDY                  &                                                               &                                                                 &          & \citep{arcomano_machine_2020}    \\
           & WRF North America       &                                                               &                                                                 &          & \citep{maulik_efficient_2022}    \\
           & ERA5                    &                                                               &                                                                 &          & \citep{dueben_challenges_2018}   \\
           & ERA5                    &                                                               &                                                                 &          & \citep{weyn_can_2019}            \\
           & ERA5                    &                                                               &                                                                 &          & \citep{weyn_improving_2020}      \\
           & ERA5                    &                                                               &                                                                 &          & \citep{weyn_sub-seasonal_2021}   \\
           & ERA5                    &                                                               &                                                                 &          & \citep{rasp_weatherbench_2020}   \\
           & ERA5                    &                                                               &                                                                 &          & \citep{rasp_data-driven_2021}    \\
           & ERA5                    &                                                               &                                                                 &          & \citep{keisler_forecasting_2022} \\
           & ERA5                    &                                                               &                                                                 &          & \citep{pathak_fourcastnet_2022}  \\
\hline
Ocean      & QG                      &                                                               &                                                                 &          & \citep{agarwal_comparison_2021}  \\
           & Shallow Water Equations &                                                               &                                                                 &          & \citep{chen_predicting_2021}     \\
           & Idealized Global GCM    &                                                               &                                                                 &          & \citep{furner_sensitivity_2022}  \\
           & Realistic Global GCM    & SST                                       &                                                                 &          & \citep{nadiga_reservoir_2021}    \\
\end{tabular}


}
\end{table}


The surrogate models described in \cref{table:all_the_emulators} show a
rapid progression of development, pushing to finer temporal and horizontal grid
resolution in a relatively short time frame.
However, even though the grid resolution of the emulators has increased, it is
not clear that the neural networks faithfully represent the scales of motion
that would be resolved by a general circulation model (GCM) at the same
resolution.
To make the discussion concrete, we present a sample of our own surrogate model
in \cref{fig:gom_sst}.
The panels show the time evolution of Sea Surface
Temperature (SST) in the Gulf of Mexico (GoM) at 1/25$^\circ$ horizontal resolution,
using data from a Navy/HyCOM 3D-Var reanalysis product \red{SECTION Y} as
``Truth'' (upper row).
We generate the prediction (middle row) with a Recurrent Neural Network (RNN) architecture
described more fully in \red{SECTION X}.
Generally speaking, the RNN captures the largest scales of SST variability over
a 36~hour window.
However, as time progresses, the SST pattern becomes overly smooth;
the RNN is
unable to capture the spatial details that are well resolved in the reanalysis
dataset, with the largest errors evolving along sharp SST fronts.

There are a number of reasons for this smoothing behavior to manifest in the
predictions, and our primary goal is to explore why this occurs.
In this work, we probe the following questions more precisely:
\begin{itemize}
    \item What spatial scales can be resolved by neural network emulators?
    \item How do fundamental choices in the training data, like temporal
        subsampling or spatial scaling, impact the prediction skill?
    \item How do architectural changes to the network impact prediction skill?
\end{itemize}
In the study, we use two forms of RNNs to emulate dynamics relevant to
geophysical fluids: Reservoir Computing and a form of Nonlinear Vector
Auto-Regression that is motivated by the Reservoir Computing paradigm (described
in \red{SECTION 2}).
We focus our attention on how well these RNNs can emulate Surface
Quasi-Geostrophic (SQG) motions \red{SECTION 2}, rather than the GoM ocean dynamics shown in
\cref{fig:gom_sst}.
By using the SQG model, we are able to quantify the resolved scales of motion more
precisely while still encountering the general behavior observed with the GoM
example.
Additionally, because we have access to the SQG model, we are able
to change the training data generation process and show the impact on prediction
skill.
In \red{SECTION 3} we using the RNN emulators of SQG dynamics to address the
questions outlined above, and we discuss the broader implications of these
results in \red{SECTION 4}.


\begin{figure}
    \centering
    \includegraphics[width=.8\textwidth]{../figures/rc_gom_sst.pdf}
    \caption{A sample prediction of SSTs in the Gulf of Mexico at 1/25$^\circ$
        horizontal resolution.
        The upper row (Truth) shows the evolution of unseen test data from the
        Navy/HyCOM reanalysis product, and the middle row shows a prediction
        from the Reservoir Computing architecture described in \red{SECTION 2}.
        The bottom row (Error) shows the absolute value of the difference between the two.
        See \red{SECTION X} for a description of the training and testing data
        used.
    }
    \label{fig:gom_sst}
\end{figure}



\section{Surface Quasi-Geostrophic Turbulence}
\label{sec:sqg}

Our goal in this study is to emulate turbulent motions relevant to realistic
geophysical fluid dynamics, while avoiding the complications
associated with the data assimilation system choices required for reanalyses
and the intricate multivariate interactions inside atmosphere or ocean GCMs.
Therefore, we aim to emulate a numerical model for
SQG turbulence as outlined by \citet{tulloch_note_2009}.
The model is formulated to represent the nonlinear Eady problem
\citep{eady_long_1949}, following \citet{blumen_uniform_1978-1}.
The model simulates turbulence
on an $f$ plane with uniform stratification and shear, bounded by rigid
surfaces $H=10$~km apart.
The motion is determined entirely by temperature advection on the boundaries
$z=\{0\text{~km},10\text{~km}\}$ as follows,
\begin{equation*}
    \dfrac{\partial \hat{\theta}}{\partial t} +
    \hat{J}(\hat{\psi}, \hat{\theta}) + ik\left(U \hat{\theta} +
        \hat{\psi}\dfrac{\partial \Theta}{\partial y}\right)
    = 0 \qquad z = 0, 10\,\text{km} \, ,
\end{equation*}
where $z=0$~km is the surface layer of the atmosphere, and $z=10$~km is
approximately at the top of the troposphere.
Here, hatted variables denote spectral components, $\hat{J}$ is the Jacobian in
spectral space, and the temperature streamfunction is
\begin{equation*}
    \hat{\psi}(z,t) = \dfrac{H}{\mu\sinh\mu}
    \left[ \cosh\left(\mu\dfrac{z}{H}\right) \hat{\theta}(H,t)
        - \cosh\left(\mu\dfrac{z-H}{H}\right) \hat{\theta}(0,t)
    \right]\, ,
\end{equation*}
with $\mu = |\mathbf{K}| NH/f$ as the nondimensional wavenumber.
We note that this model produces an approximate spectrum of
$|\mathbf{K}|^{-5/3}$ without any break (\cref{fig:sqg-reference}),
as is expected in Eady turbulence.
For more details on this model, see \citep{tulloch_note_2009}.


\begin{figure}
    \centering
    \includegraphics[width=\textwidth]{../figures/sqg_reference_plot.jpg}
    \caption{A reference snapshot from the SQG dataset. The left and middle panels
        show snapshots of potential temperature anomaly at the surface and
        top-of-troposphere layers, respectively.
        The right panel shows the kinetic energy density spectrum associated
        with this snapshot (black line), compared to
        $|\mathbf{K}|^{-5/3}$ (dashed line).
    }
    \label{fig:sqg-reference}
\end{figure}

The model is discretized in space with $N_x = N_y = 64$ and $\nvertical=2$,
uses a periodic boundary in both horizontal directions,
and uses a timestep of $\Delta t=5$~minutes.
To generate datsets for the neural networks, we initialize the model with Gaussian i.i.d.
noise and spinup for 360~days, which we define as one model year.
The spinup period is discarded, and we then generate a 25~year dataset that we partition
into training (first 15~years), validation (following 5~years), and testing
(final 5~years).
For validation and testing, we randomly select 12~hour time windows from each
respective dataset.

\section{Methods}
\label{sec:methods}

\subsection{Datasets Used}
\label{subsec:datasets}

For each discuss what it simulates, how accessed/generated, how much is used for
training and validation, and the scale/size of the dataset
\begin{itemize}
    \item Simple QG / MAOOAM
    \item SQG Turbulence
    \item GoM Reanalysis
\end{itemize}


\subsection{Recurrent Neural Network Architecture}
\label{subsec:rnn-architecture}

Reservoir computing in general:
\begin{itemize}
    \item equations
    \item hyperparameters
    \item some interpretation of what's going on
\end{itemize}

Parallelization scheme, with a figure.

\subsubsection{Training}
\label{subsubsec:training}

Cost function, minimization, input signal drives reservoir, synchronization

\subsubsection{Validation}
\label{subsubsec:validation}

\begin{itemize}
    \item recurrence relation in RC forecasting
    \item how many samples used
    \item VPT metric
    \item skill benchmarks: persistence, AR1, (maybe) coarse model
\end{itemize}

\subsection{NVAR Results}
\label{subsec:nvar-results}


\subsubsection{Temporal Subsampling}


\begin{figure}
    \centering
    \includegraphics[width=\textwidth]{../figures/nvar_big_plot.jpg}
    \caption{One sample NVAR prediction from the test dataset for $\nsub =
        1,4,16$, shown in the second, third, and fourth rows.
        The corresponding truth is shown in the top row.
        In order to emphasize the changing features in the plot, each panel
        shows the difference between the state after 1.33, 4, and 8 hours with
        the initial conditions ($t_0$), corresponding to the left, middle, and
        right columns.
        Here $\maxlag=1$.
    }
    \label{fig:nvar_qualitative}
\end{figure}

% Describe the big plot
% diff btwn t0, Nlag=1, one sample from test dataset
\cref{fig:nvar_qualitative} shows a qualitative comparison of NVAR predictions
as a function of $\nsub$, i.e., how frequently the training data are sampled and
the model makes predictions.
Each panel shows the difference between a given timestep and the initial
conditions in order to emphasize the features that change over an 8~hr window.
The top row shows the truth, while subsequent rows show the evolution of NVAR
predictions for $\nsub=1,4,16$.
For this figure, we set $\nlag=1$.

% Describe qualitatively what happens as we subsample the data further
% At Nsub=1, unstable, some smoothness especially at longer times but less so at
% short times
% As Nsub increases, stable, but "more smooth" at small scales even at shorter
% prediction windows
At the model timestep ($\Delta t = $5~min; $\nsub=1$), the NVAR predictions are
qualitatively similar to the truth for relatively short forecast windows.
That is, many of the fronts and dipoles that exist in the truth are also evident
in the predictions.
However, at longer forecast windows the predictions are unstable.
In this specific sample, instabilities are present after 8~hours have passed.
Additionally, as the forecast develops, the prediction becomes somewhat smoother
than the truth, which is evident as many small scale features are connected in
the prediction.
As the data are subsampled at lower frequencies, the predictions are stable, but
show a more dramatic smoothing effect.
For instance, at $\nsub=16$ ($\Delta t = $1.33~hrs), the smoothing effect is
evident even after a single timestep (i.e., the first column in
\cref{fig:nvar_qualitative}),
as many small scale features are broader than the truth and fronts generally
exhibit lower amplitudes.

\begin{figure}
    \centering
    \includegraphics[width=\textwidth]{../figures/nvar_big_ke_relerr.pdf}
    \caption{NVAR Spectra}
    \label{fig:nvar_spectra}
\end{figure}

% To show a quantitative discussion, consider the relative error in the KE density
% First column only: colors show error at different time stamps, spread
% indicates error over 50 samples in test dataset,
% In all plots, error is largest at the small spatial scales, corresponding to
% larger wavenumbers
% Note that not all of the Nsub=1,4 sims are stable, so the relative error is
% unacceptably large for t0+8 hrs, well beyond what is shown...
This smoothing behavior is captured quantitatively in \cref{fig:nvar_spectra},
which shows the relative error of the Kinetic Energy Density spectrum
coefficients as a function of forecast time, $\nsub$, and $\nlag$.
The solid lines indicate average error and the shading indicates a spread of one
standard deviation, computed based on 50~sample predictions each initialized
from different initial conditions in the test dataset.
Generally speaking, error is largest in all cases at smaller spatial scales,
which corresponds to the larger wavenumbers.
%Note also that not all of the predictions produced with $\nsub=\{1,4\}$ are
%stable, so the relative error is \red{unacceptably large} after 8~hours, and
%therefore not shown.

% Nsub=1,4 produce nearly identical errors at small scales for short lead times,
% Nsub 1 generates instabilities in many simulations early on, and so error is
% large
% by subsampling the data more dramatically, Nsub=16, we see the error in the
% small scales grows considerably (by at least 40%).
At short forecast time horizons, the relative error is smallest and nearly
identical for the high frequency simulations: $\nsub=\{1,4\}$.
However, these predictions generate instabilities rapidly, in some cases as
early as after 4~hours for $\nsub=1$, and so the relative error is
\red{large, i.e., $>>1$ by 8~hours}.
By increasing the subsampling factor to $\nsub=16$, the simulations are stable.
However, the relative error 2-3 times larger for wavenumbers
$> 2\cdot10^{-3}$~rad~km$^{-1}$.
This error at higher wavenumbers corresponds to the qualitatively smooth
prediction shown in \cref{fig:nvar_qualitative}, as the small spatial scale
features are simply not resolved.


\subsubsection{Prediction Skill as a Function of Memory}

% A key feature of RNNs and autoregressive models is memory
% NVAR gives us the opportunity to explore this explicitly.
A key feature of RNNs and autoregressive models is that they retain memory of
previous system states.
We explore the effect of adding memory within the NVAR architecture explicitly
by increasing $\nlag$, the largest number of lagged states used to create the
feature vector.
The relative error is shown in \cref{fig:nvar_spectra}, for
$\nlag=\{1,2,\text{ and }3\}$ in
the left, middle, and right columns, respectively.
Broadly, increasing $\nlag$ shows similar behavior to reducing $\nsub$:
the error at short time horizons is reduced, especially at wavenumbers
$\ge 2\cdot10^{-3}$~rad~km$^{-1}$,
but grid scale noise and unphysical instabilities
are generated more readily.

At 1.33~hours, increasing $\nlag$ reduces the error at all wavenumbers for
$\nsub=1$, and at all but the largest wavenumber for $\nsub=4$.
With these two subsampling factors, error explodes at wavenumbers
$>4\cdot10^{-3}$~rad~km$^{-1}$,
corresponding to unphysical instabilities similar to what is
shown in \cref{fig:nvar_qualitative} \todo{label and reference panels}.
With higher subsampling, $\nsub=16$, error is reduced by as much as 20\% at
$t=t_0 + 1.33$~hrs.
However, as time passes, improvements at small scales
($>2\cdot10^{-3}$~rad~km$^{-1}$) is more muted, while error actually grows
with $\nlag$ at the larger spatial scales.
\todo{why large error at large scales?}

\begin{figure}
    \centering
    \includegraphics[width=.5\textwidth]{../figures/nvar-mrmse-vs-lag-050samples.pdf}
    \caption{Minimum RMSE}
    \label{fig:nvar_min_rmse}
\end{figure}

The behavior discussed here is summarized by the boxplots in
\cref{fig:nvar_min_rmse}, which shows the minimum RMSE achieved by NVAR for a
given $\nsub$ (x-axis) and $\nlag$ (color).
The best prediction skill, which as we noted is achieved within the first
1-2~hours of the forecast, is best for smaller $\nsub$.
Increasing $\nlag$ can improve this prediction skill to make up for subsampled
data, albeit with dimishing returns.
Additionally, for $\nsub$ large enough
(here $>4$), the gains in ``best possible'' skill do not approach what can be
achieved with $\nsub\le 4$, presumably due to the fact that small scale features
cannot be resolved when enough temporal information is skipped.

\section{Echo State Network Prediction Skill}
\label{sec:esn-results}

In this section we show the prediction skill of the more general ESN architecture outlined in \cref{subsec:rc}.
Here we use similar metrics as in \cref{sec:nvar-results} to evaluate the ESN
skill, except that we show time averaged quantitative metrics because all of the
ESN predictions are stable for the full twelve-hour forecast horizon.
That is, when shown as a single distribution rather than a time series, NRMSE is reported as
\begin{linenomath*}\begin{equation}
    \text{NRMSE} = \sqrt{
            \dfrac{1}{\ntime\nstate}\sum_{n=1}^{\ntime}\sum_{i=1}^{\nstate}\left(
        \dfrac{\hat{v}_i(n) - v_i(n)}{SD}
        \right)^2 } \, ,
    \label{eq:total-nrmse}
\end{equation}\end{linenomath*}
where $\ntime$ consists of the number of timesteps in the trajectory.
In order to characterize spectral error, we show the KE relative error as in
\cref{sec:nvar-results}.
Additionally, we show the NRMSE in terms of the KE density spectrum as follows
\begin{linenomath*}\begin{equation}
    \text{KE\_NRMSE} = \sqrt{
            \dfrac{1}{\ntime\nk}\sum_{n=1}^{\ntime}\sum_{k=1}^{\nk}\left(
            \dfrac{\hat{E}(n, k) - E(n, k)}{SD(k)}
            \right)^2} \, ,
    \label{eq:ke_nrmse}
\end{equation}\end{linenomath*}
where $\nk$ is the number of spectral coefficients and $SD(k)$ is the temporal
standard deviation of each spectral coefficient throughout the test trajectory.
As in \cref{sec:nvar-results}, all distributions and lineplots indicate
prediction skill from 50 randomly selected initial conditions from an unseen
test dataset.


\subsection{Soft Constraints on Spectral Error}
\label{subsec:esn-ego}

% Here we present ESN results, using mainly the two metrics used to evaluate
% NVAR
It is well known that ESN prediction skill is highly dependent on the global or
``macro-scale'' parameters noted in
\cref{eq:rc-hyperparameters},
\citep<$\esnparams$, e.g.>[]{platt_systematic_2022,lukosevicius_practical_2012}.
Following the success of previous studies in using Bayesian Optimization methods
to systematically tune these parameters
\citep{griffith_forecasting_2019,penny_integrating_2022,platt_systematic_2022},
we use the Bayesian Optimization algorithm outlined by \citet{jones_efficient_1998} and implemented by
\citet{bouhlel_python_2019} to find optimal parameter values.

More recently, \red{Platt et al.}\todo{Cite Jason}
showed that constraining these macro-scale
parameters using global invariant properties of the underlying system leads the
optimization algorithm to select parameters that generalize well to unseen test data. In that work, the authors were successful in using the largest positive
Lyapunov exponent, and to a lesser extent the fractal dimension of the system.
Because of the focus on resolved scales in this work, we take a similar approach, but test the effect of constraining the ESN to the KE density spectral coefficients.
Specifically, we implement the following two-stage training process.
At each step, the macro-scale parameters, $\esnparams$, are fixed, and the
``micro-scale'' parameters $\Wout$ are obtained by minimizing \cref{eq:cost}.
This readout matrix is then used to make forecasts from randomly selected
initial conditions from a validation dataset.
The skill of each of these forecasts is captured by the macro-scale cost
function
\begin{linenomath*}\begin{equation}
    \cf_\text{macro}(\esnparams) = \dfrac{1}{\nmacro}
    \sum_{j=1}^{\nmacro}
    \left\{
        \text{NRMSE}(j) + \gamma \text{KE\_NRMSE}(j)
    \right\}
    \label{eq:macro-cost} \, ,
\end{equation}\end{linenomath*}
where NRMSE and KE\_NRMSE are defined in \cref{eq:total-nrmse,eq:ke_nrmse},
$\nmacro$ is the number of forecasts used in the validation set, and $\gamma$ is
a hyperparameter that determines how much to penalize deviations
from the true KE density spectrum.
The value of $\macrocost$ is then used within the Bayesian Optimization algorithm, which reiterates the whole optimization process with new values for
$\esnparams$ until an optimal value is found or the maximum number of iterations is reached.
Here, we use $\nmacro=10$, initialize the optimization with 20~randomly sampled points in the 5~dimensional parameter space, and run for 10~iterations. Note that we run this optimization procedure for each unique ESN configuration
throughout \cref{sec:esn-results} (i.e., for each $\nsub$ and each $\gamma$
value).

\cref{fig:rc_qualitative_nsub01} shows a qualitative view of how penalizing the
KE density impacts ESN prediction skill when it operates at the original
timestep of the SQG model (i.e., $\nsub=1$).
At $\gamma=0$, the ESN parameters are selected based on NRMSE alone, and
the prediction is relatively blurry.
However, as $\gamma$ increases to $10^{-1}$, the prediction becomes sharper as
the small scale features are better resolved.

\cref{fig:rc_quantiative_nsub01} gives a quantitative view of how the KE density
penalty changes ESN prediction skill, once again with $\nsub=1$.
The first two panels show that there is a clear tradeoff between NRMSE and KE error:
as $\gamma$ increases the NRMSE increases but the spectral representation improves.
The final panel in \cref{fig:rc_quantiative_nsub01}
shows that the spatial scales at which the spectral error manifests in these
different solutions.
When $\gamma=0$, the macro-scale parameters are chosen to minimize NRMSE,
leading to blurry predictions and a dampened spectrum at the higher wavenumbers,
especially for $|\mathbf{K}| > 2\cdot10^{-3}$~rad~km$^{-1}$.
We note that \citet{lam_graphcast_2022} report the same behavior when using a cost function that is purely based on mean-squared error.
On the other hand, when $\gamma = 10^{-1}$, the global parameters are chosen to
minimize both NRMSE and KE density error, where the latter treats all spatial
scales equally.
In this case, KE relative error is reduced by more than a factor of two and the
spectral bias at higher wavenumbers is much more muted.
We note that using larger values of $\gamma$ produces similar results to
$\gamma=10^{-1}$.

Of course, the tradeoff for the reduced spectral error is larger NRMSE, resulting
from slight mismatches in the position of small scale features in the forecast.
However, our purpose is to generate forecasts that are as representative
of the training data as possible.
Overly smoothed forecasts are not desirable, because this translates to losing local extreme values,
which are of practical importance in weather and climate.
Additionally, a key aspect of ensemble forecasting is that the truth remains a
plausible member of the ensemble \citep{kalnay_ensemble_2006}.
Therefore, representing the small scale processes, at least to some degree,
will be critical for integrating an
emulator into an ensemble based prediction system.

Finally, we note that there is some irreducible high wavenumber error,
which is most clearly seen by comparing the prediction skill to a persistent
forecast.
While the sample median NRMSE for each $\gamma$ value beats persistence, the
KE\_NRMSE is more than double, due to this error at the small spatial scales.
Ideally, our forecasts would beat persistence in both of these metrics, but
obtaining the ``realism'' in the small spatial scales necessary to
dramatically reduce this spectral error should be addressed in future work.

%However, as time progresses through the 12~hour forecast window,
%the KE relative error increases slightly at the larger spatial scales,
%$|\mathbf{K}| \lessapprox 10^{-3}$~rad~km$^{-1}$ (not shown),
%and this error is what causes NRMSE to be roughly 10\% higher in the case where
%$\gamma=10^{-1}$ than $\gamma=0$.

\begin{figure}
    \centering
    \includegraphics[width=\textwidth]{../figures/rc_qualitative_gamma.jpg}
    \caption{
        One sample prediction from the test dataset, where each panel shows
        potential temperature in the truth (left) and subsequently for
        ESN predictions with parameters optimized using
        $\gamma$~=~\{0,~$10^{-2}$,~$10^{-1}$\} in \cref{eq:macro-cost}.
        Each panel shows the prediction at a forecast lead time of 4~hours,
        using the same initial conditions as in \cref{fig:nvar_qualitative}.
        As $\gamma$ increases from left to right, the prediction becomes sharper
        (i.e., less blurry).
        Here, the ESN is evaluated at the SQG model timestep, i.e., $\nsub$~=~1.
    }
    \label{fig:rc_qualitative_nsub01}
\end{figure}

\begin{figure}
    \centering
    \includegraphics[width=\textwidth]{../figures/rc_all_nsub01.pdf}
    \caption{
        Quantitative comparison of ESN predictions at $\nsub$~=~1 with macro-scale
        parameters
        chosen using different values of $\gamma$ in \cref{eq:macro-cost}.
        NRMSE (\cref{eq:total-nrmse}; left), KE\_NRMSE (\cref{eq:ke_nrmse};
        middle), and
        KE relative error (\cref{eq:ke_relerr}; right) highlight the tradeoff
        between minimizing NRMSE and spectral error: as $\gamma$ increases
        spectral error is reduced, but NRMSE increases.
        Note that the KE relative error is shown at 4~hours to provide
        direct comparison to the snapshots in \cref{fig:rc_qualitative_nsub01}.
        In each plot, the solid gray line indicates the median skill of a persistent
        forecast.
    }
    \label{fig:rc_quantiative_nsub01}
\end{figure}



\subsection{Temporal Subsampling}
\label{subsec:esn-subsampling}


The NVAR predictions shown in \cref{subsec:nvar-subsampling} indicate that
subsampling the training data systematically increases error at small spatial
scales.
However, the architecture was not specifically designed or constrained to
have a good spectral representation of the underlying dynamics.
On the other hand, the previous section (\cref{subsec:esn-ego})
showed that the spectral bias at high wavenumbers
can be reduced by optimizing the global ESN parameters
to the true KE density spectrum.
Given these two results, we explore the following question: does temporal subsampling still
increase spectral bias in the more general ESN framework, even when parameters
are chosen to minimize this bias?

\cref{fig:rc_qualitative_gamma0.1} and \cref{fig:rc_quantiative_gamma0.1}
show that even when the macro-scale parameters are chosen to prioritize the KE
density representation (i.e., $\gamma = 10^{-1}$ is fixed),
temporal subsampling does lead to an apparently inescapable spectral bias.
This effect is shown qualitatively in \cref{fig:rc_qualitative_gamma0.1},
where the predictions become
smoother as the temporal subsampling factor, $\nsub$, increases.
The effect is similar to what was seen with NVAR except the blurring effect is
less pronounced.
Quantitatively, \cref{fig:rc_quantiative_gamma0.1}(b) shows that as $\nsub$
increases, error in KE density spectrum generally increases, while panel (c) shows that this KE
error is concentrated in the small spatial scales,
$|\mathbf{K}| > 2\cdot10^{-3}$~rad~km$^{-1}$.
We note that the degree of spectral bias at $\nsub=16$ is smaller than what was
achieved with NVAR for the same $\nsub$ value, cf. \cref{fig:nvar_ke_vs_lag},
indicating that the optimization was successful in reducing the spectral bias.

\begin{figure}
    \centering
    \includegraphics[width=\textwidth]{../figures/rc_qualitative_nsub.jpg}
    \caption{One sample prediction from the test dataset, exactly as in
        \cref{fig:rc_qualitative_nsub01}, except here $\gamma$~=~$10^{-1}$ is fixed, and
        the temporal subsampling factor is varied: $\nsub$~=~\{1,~4,~16\}.
        As the temporal subsampling factor increases, the small spatial scale
        features are lost and the prediction becomes blurrier.
    }
    \label{fig:rc_qualitative_gamma0.1}
\end{figure}

\begin{figure}
    \centering
    \includegraphics[width=\textwidth]{../figures/rc_all_gamma0.1.pdf}
    \caption{Quantitative comparison of ESN predictions, showing
        NRMSE (left), KE\_NRMSE (middle), and KE relative error (right), exactly as in
        \cref{fig:rc_quantiative_nsub01}, except here $\gamma$~=~$10^{-1}$ is fixed,
        and the temporal subsampling factor is varied: $\nsub$~=~\{1,~4,~16\}.
        As the temporal subsampling factor increases, spectral errors increase.
        In each plot, the solid gray line indicates the median skill of a persistent
        forecast.
    }
    \label{fig:rc_quantiative_gamma0.1}
\end{figure}

Interestingly, there is little difference between NRMSE obtained by the ESNs at
different $\nsub$ values.
Additionally, \cref{fig:rc_quantiative_gamma0.0} shows that there is little
difference in both NRMSE and KE\_NRMSE when $\gamma=0$, i.e., when NRMSE is the only criterion
for parameter selection.
This result shows that NRMSE alone is not a good criterion for model selection, given
that we have shown success in reducing spectral errors by prioritizing the spectrum appropriately.

\begin{figure}
    \centering
    \includegraphics[width=\textwidth]{../figures/rc_all_gamma0.0.pdf}
    \caption{Same as \cref{fig:rc_quantiative_gamma0.1}, except here $\gamma$~=~0,
        indicating that only NRMSE is penalized in the cost function.
        The error is relatively similar, indicating that NRMSE alone is a
        suboptimal penalty for model selection.
        In each plot, the solid gray line indicates the median skill of a persistent
        forecast.
    }
    \label{fig:rc_quantiative_gamma0.0}
\end{figure}

\subsection{Impact of the Hidden Layer Dimension}
\label{subsec:esn-size}

The dimension of the hidden layer, $\nhidden$, also known as the reservoir size, determines the
memory capacity available to the ESN
\citep{jaeger_echo_2001,lukosevicius_practical_2012}.
For systems with high dimensional input signals, it is crucial to use a sufficiently large hidden layer to afford the memory capacity necessary for accurate
predictions \citep{hermans_memory_2010}.
In all of the preceding sections we fixed $\nhidden=6,000$ for each local group,
where for reference each local group has an input dimension of
$\nlocalinputstate=200$ and an output dimension of $\nlocalstate=128$.
Here, we briefly address the effect of doubling the hidden layer dimension, while keeping the input and output dimensions constant, in order to test how
sensitive our conclusions are on this crucial hyperparameter.
Due to the computational expense of the parameter optimization discussed in
\cref{subsec:esn-ego}, we only perform this experiment for $\nsub=16$.

The impact of doubling $\nhidden$ on prediction skill is shown in
\cref{fig:esn-size}, where for the sake of brevity we only show results for the
case when $\gamma=10^{-1}$ in \cref{eq:macro-cost}.
The left panel shows that the larger hidden layer actually increases the NRMSE slightly.
However, the middle and right panels show that this increase is due to the improved
spectral representation.
The improvement in KE\_NRMSE is nearly proportional to the improvement achieved by increasing
the temporal resolution of the data.
That is, doubling the hidden layer width reduces the average KE\_NRMSE by 14\%,
while increasing the temporal resolution of the data by a factor of 4 reduces
the KE\_NRMSE by 30\%.
These results indicate a potential brute force approach to overcoming the
subsampling related spectral errors.
However, the larger hidden layer dimension has to be constrained
with enough training data, and requires more
computational resources.

\begin{figure}
    \centering
    \includegraphics[width=\textwidth]{../figures/rc_reservoir_size.pdf}
    \caption{The impact of doubling the hidden layer dimension from
        $\nhidden$~=~6,000 to
        $\nhidden$~=~12,000 on NRMSE (left), KE\_NRMSE (middle), and KE relative
        error (right).
        Increasing the hidden layer dimension is relatively proportional to reducing
        the temporal subsampling factor, indicating a potential brute force
        approach to reducing the subsampling related spectral errors.
        Here $\gamma$~=~$10^{-1}$, and the solid gray line indicates the
        median skill of a persistent forecast.
    }
    \label{fig:esn-size}
\end{figure}

\subsection{Impact of Training Dataset Size}
\label{subsec:esn-fixed-steps}

In all of the preceding experiments, the length of training time was fixed to
15~years, meaning that there are fewer training samples when the data are
subsampled, i.e., as $\nsub$ grows.
Specifically, 15~years of data at an original model timestep of 5~minutes means
that there are approximately
$1.6\cdot10^{6}$, $3.9\cdot10^5$, and $9.72\cdot10^4$ samples
for each case previously shown: $\nsub=1$, 4, and 16, respectively.
Here, we show that even when the number of training samples is fixed, the
subsampling related spectral errors are still present.

\cref{fig:esn-fixed-steps} shows the prediction skill in terms of NRMSE and
spectral errors when the number of training samples is fixed to $9.72\cdot10^4$.
With this number of samples, the training data is exactly the same for
$\nsub=16$, but only spans $3.75$ and $0.94$~years for $\nsub=4$ and $\nsub=1$,
respectively.
However, we see the same general trend as before: subsampling the data improves
NRMSE slightly but increases the KE\_NRMSE.
As before, the spectral error is largest in the higher wavenumbers,
$|\mathbf{K}| > 2\cdot10^{-3}$~rad~km$^{-1}$.
We note that the difference in performance between $\nsub=4$ and $\nsub=16$ is
marginal.
The only notable difference between these two cases is that the ESN is less
consistent, i.e., the KE\_NRMSE distribution is broader, when $\nsub=16$.
However, it is clear that spectral error is lowest when the data are not
subsampled at all, even though less than a year of data is used.
This result indicates that there could be a benefit to training a RNN on a
relatively shorter model trajectory that is untouched, rather than a longer
dataset that is subsampled in time.

\begin{figure}
    \centering
    \includegraphics[width=\textwidth]{../figures/rc_fixed_steps.pdf}
    \caption{Subsampling related spectral errors persist even when the number of
        training samples is fixed. Here, the number of samples is fixed to
        $9.72\times10^{4}$ for all cases, and yet the temporal subsampling
        related spectral errors remain.
        Here, $\gamma$~=~$10^{-1}$ and the solid gray line indicates the median skill of a persistent
        forecast.
    }
    \label{fig:esn-fixed-steps}
\end{figure}

\section{Discussion}
\label{sec:discussion}

Lots of things to discuss.


\appendix
\section{Unique Architecture Changes}
\label{sec:new_methods}

Here we describe several aspects of our neural network implementations that are
unique with respect to previous Reservoir Computing works.
Generally speaking, there were a number of architectural details that we tested
while developing this code, many of which were covered by
\citet{platt_systematic_2022}.
Here we cover a few more that turned out to be important for the larger-scale
SQG system.
In general, we provide empirical evidence for these choices using the Lorenz96 dataset described
in \red{SECTION X}.
Of course, these tests are insufficient to definitively prove that these choices
will translate perfectly to the SQG system.
However, we consider this as a bare minimum test that will catch downright bad
design choices, while saving the computing resources necessary to train an
emulator for SQG turbulence.

Additionally, we note that our motivation to discuss these details stems from
the fact that it is common for reservoir computing to be
implemented from scratch, due to its simplicity, as opposed to other neural network architectures which
often make use of existing libraries like Tensorflow or PyTorch \red{CITE}.
Therefore, even though some of these details, especially those discussed in
\cref{subsec:data_normalization} and \cref{subsec:data_misfit}, may seem
obvious to scientists familiar with more advanced neural networks, we point them
out as a reference to anyone else using these methods.


\subsection{Input Matrix Scaling}

The first is how we form the input matrix $\inputmatrix$.
Typically, $\inputmatrix$ is filled with entries
\begin{equation*}
    w_{i,j} \sim \mathcal{U}(-\sigma,\sigma) \qquad
    i = [1, 2, ..., \nhidden], j=[1,2, ..., \ninputstate] \,
\end{equation*}
where $\sigma$ is a hyperparameter that determines the bounds of the uniform
distribution.
Here we found it to be more advantageous normalize the input matrix by the
largest singular value.
That is, we first compute $\hat{\mathbf{W}}_\text{input}$, with elements
\begin{equation*}
    \hat{w}_{i,j} \sim \mathcal{U}(-1,1) \qquad
    i = [1, 2, ..., \nhidden], j=[1,2, ..., \ninputstate] \, .
\end{equation*}
Then, we set $\inputmatrix$ as
\begin{equation*}
    \inputmatrix \coloneqq
    \dfrac{\sigma}{\sigma_{max}\left(\hat{\mathbf{W}}_\text{in}\right)}
    \hat{\mathbf{W}}_\text{in} \,
\end{equation*}
where $\sigma_{max}\left(\cdot\right)$ is the largest singular value, and
the hyperparameter $\sigma$ is the desired largest singular value of
$\inputmatrix$.
Our motivation for using this type of normalization is that we found it
necessary to use very wide hyperparameter optimization bounds for $\sigma$ when
using the standard input scaling strategy.
Using the largest singular value balances the fact that
the amplitude of the contributions to the reservoir, i.e., the elements of the
vector
\begin{equation*}
    \mathbf{p} = \inputmatrix \inputstate =
    \begin{pmatrix}
        \mathbf{w}_1^T\inputstate \\
        \mathbf{w}_2^T\inputstate \\
        \vdots \\
        \mathbf{w}_{\nhidden}^T\inputstate
    \end{pmatrix}
\end{equation*}
grow with $\ninputstate$.
By controlling for this growth, we were able to reduce the optimization bounds
and started achieving more sensible results.

Additionally, we note that it appears advantageous to use this normalization
strategy even for small systems.
\red{Figure X} shows the valid prediction time achieved with the \red{6D
Lorenz96 system described in Section Y}.
The histograms titled ``baseline'' and ``sv\_input'' denote the difference
between the usual input scaling method and the method discussed here,
respectively.
We see that on average, there is an improvement of \red{0.7 timescales},
accounting for different randomly generated adjacency and input matrices, and
100 random samples from the test dataset.
Therefore, we use this singular value normalization method in our SQG
experiments \red{Section Z}.

\subsection{Adjacency Matrix Scaling}

Typically, the reservoir adjacency matrix is normalized to achieve a desired
spectral radius.
That is, the matrix $\hat{\adjacency}$ is generated with elements
$\hat{a}_{i,j} \sim \mathcal{U}(-1,1)$, where $i,j$ are random indices in order
to satisfy the desired sparsity of the matrix (all other elements are 0).
Then, the $\adjacency$ is set as
\begin{equation*}
    \adjacency \coloneqq
    \dfrac{\spectralradius}{\lambda\left(\hat{\adjacency}\right)}
    \hat{\adjacency} \, ,
\end{equation*}
where $\lambda\left(\cdot\right)$ is the spectral radius, and $\spectralradius$
scales the matrix to achieve the desired spectral radius.
A common guideline is to set $\spectralradius \simeq 1$, as it is hypothesized
that this puts the reservoir on the ``edge of stability'' so that it performs
well in emulating nonlinear systems \red{CITE}.
However, as originally described by \citet{jaeger_echo_2001},
using the spectral radius is only a necessary, but insufficient means to satisfy
the Echo State Property \red{DEFINE THIS}.
On the other hand, using the largest singular value is a sufficient condition
for satisfying the echo state property.

In our experimentation, we have found more success using the largest singular
value to normalize the adjacency matrix.
\red{Figure X(b)} shows the difference between using the spectral radius and
singular value to normalize the adjacency matrix, marked as ``SR'' and
``SV'', respectively.
Over a range of randomly initialized input and adjacency matrices, and 100
samples in the test dataset, we see an improvement of \red{0.6 timescales} on
average, and so we use this normalization strategy throughout.

\subsection{Data Normalization}
\label{subsec:data_normalization}

A key aspect in machine learning is normalizing input data before passing it to
the model.
Experiments from \citet{platt_systematic_2022} showed, however, that the
standard approach to normalizing data
be detrimental to prediction skill.
By ``standard  approach'', we mean
\begin{equation*}
    u_i(t) = \dfrac{u_i(t) - \bar{u}_i}{\sigma_i} \qquad i = [1, 2, ...
    \ninputstate]
\end{equation*} \todo{Use something other than sigma for SD}
where $\bar{u}_i, \sigma_i$ are the mean and standard deviation taken from the
training data for each channel
of input data, indexed by $i$.
The key takeaway from \citet{platt_systematic_2022} is that by using separate
normalization values for each channel, the covarying relationships between the
data are destroyed and so the reservoir cannot learn the true dynamics.
They therefore propose two additional normalization methods, and we suggest one
additional scheme.
Here, we propose to simply normalize by removing the mean and normalizing by the
standard deviation taken over all channels and timesteps in the training data.
We compare this method to the slightly different method proposed by
\citet{platt_systematic_2022}, which does the same thing but normalizes by the
range $\max{\inputstate} - \min{\inputstate}$ rather than the standard
deviation.

\red{Figure X} shows a comparison of the different methods.

\subsection{Data Misfit}
\label{subsec:data_misfit}

Finally, we note that it is common in the Reservoir Computing literature to see
the cost function, \cref{eq:cost} expressed in terms of the total misfit, rather
than the time average misfit as is shown here.
In our experiments, we found it more useful to use time mean misfit
because we used a large amount of data, i.e., more than a million time stamps.
Using the time mean rather than total misfit reduced the order of magnitude
differences between the matrices $RR^T$, $RY^T$ with respect to the tikhonov
parameter $\tikhonov$.
This has the practical implication of reducing the bounds placed on $\tikhonov$
during the optimization routine, which helped improve its convergence.
\todo{This is pretty hand-wavy, and really had more of an impact on NVAR}




%During our experimentation we tested a variety of parallelization stencils with
%different local input vector sizes.
%
%
%
%Although the choice of $\sigma$ has received less attention in comparison to the
%spectral radius parameter $\spectralradius$, it serves a critical role in
%determining the amplitude of the input signal into the reservoir because it
%helps determine the region of the hyperbolic tangent activation function that
%the reservoir lies.
%If the input signal is too small, the reservoir may lie only on the linear
%region, while if it is too large, then the activation function will behave like
%a binary switch, oscillating between -1 and +1.
%
%We have found that this input scaling is especially important for large scale
%systems, i.e., $\ninputstate \sim \mathcal{O}(10^2)$ and
%$\nhidden \sim \mathcal{O}(10^{3}-10^{4})$.
%To illustrate, consider the amplitude of the signal that is received by the
%reservoir at each time step, described by the Euclidean norm:
%$\norm{\inputmatrix\inputstate}_2$.
%For sake of simplicity, if we assume that the input signal has unit norm,
%$\norm{\inputstate}_2=1$, then the amplitude received by the reservoir is
%determined by the induced norm of $\inputmatrix$.
%That is,
%\begin{equation*}
%    \norm{\inputmatrix}_2 \coloneqq
%    \sup\{ \norm{\inputmatrix \inputstate}_2 :
%        \norm{\inputstate}_2 = 1 \} \, .
%\end{equation*}
%For
%
%
%Figure X shows how the induced norm grows with the reservoir size, which
%determines the number of rows in the matrix.
%The solid line and shading show approximations from N randomly initialized
%matrices.
%Additionally, we show how the induced norm changes as a function of $\sigma$.
%The slopes are X and Y, indicating that for every N increase in the number of
%rows in the matrix, we

\section{Gulf of Mexico Dataset}
\label{sec:gom}



\newpage
\section{Open Research}
Code will be made available prior to publication.
%AGU requires an Availability Statement for the underlying data needed to understand, evaluate, and build upon the reported research at the time of peer review and publication.
%
%Authors should include an Availability Statement for the software that has a significant impact on the research. Details and templates are in the Availability Statement section of the Data and Software for Authors Guidance: \url{https://www.agu.org/Publish-with-AGU/Publish/Author-Resources/Data-and-Software-for-Authors#availability}
%
%It is important to cite individual datasets in this section and, and they must be included in your bibliography. Please use the type field in your bibtex file to specify the type of data cited. Some options include Dataset, Software, Collection, ComputationalNotebook. Ex:

%%%%%%%%%%%%%%%%%%%%%%%%%%%%%%%%%%%%%%%%%%%%%%%

\acknowledgments

T.A. Smith and S.G. Penny acknowledge support from NOAA grant NA20OAR4600277.
S.G. Penny and J.A. Platt acknowledge support from the Office of Naval Research
(ONR) grants N00014-19-1-2522 and N00014-20-1-2580. T.-C. Chen is supported by
the NOAA Cooperative Agreement with CIRES, NA17OAR4320101.


%% ------------------------------------------------------------------------ %%
%% References and Citations

%%%%%%%%%%%%%%%%%%%%%%%%%%%%%%%%%%%%%%%%%%%%%%%
%
% don't specify bibliographystyle
\bibliography{references}
\end{document}
