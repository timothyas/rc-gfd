\section{Introduction}
\label{sec:intro}

\begin{itemize}
    \item Introduce machine learning for geo/earth/climate/weather systems
        generally. Lots of available data these days from model runs (e.g. CM
        runs from GFDL, CMIP for climate) and data-constrained reanalyses.
        Combine this with
        improvements in processing power (GPUs, memory, ...), data-driven
        approaches being explored for a number of purposes (e.g.\ emulation,
        parameterization, and process identification)
    \item Focus in on the emulation problem, and why it is a worthwhile pursuit.
    \item Discuss past work, noting that it's usually at low resolution, including:
        \begin{itemize}
            \item Ryan Keisler's graph neural network
            \item Pathak and NVIDIA fourcastnet
            \item Arcomano emulator and hybrid approach
                \cite{arcomano_machine_2020}
        \end{itemize}
    \item Here we focus on:
        \begin{itemize}
            \item developing high resolution emulators
            \item RNNs for temporal information
            \item focus on good deterministic skill (find cite to
                back up better emulator = more likely to better represent
                stats, which is our end goal)
            \item Focus is general: want high resolution GFD model for NWP,
                ocean forecasting, renalysis, coupled modeling (component replacement).
        \end{itemize}
\end{itemize}
