\section{Conclusions}
\label{sec:conclusions}

Recent advances in neural network based emulators of Earth's weather and climate
indicate that forecasting centers could benefit greatly from incorporating
neural networks into their future prediction systems.
However, a common issue with these data-driven models is that they produce
relatively blurry predictions, and misrepresent the small spatial scale features
that can be resolved in traditional, physics-based forecasting models.
Here, we showed that the simple space saving step of subsampling the training
data used to generate recurrent neural network emulators accentuates this
small scale error.
While we show some success in mitigating the effects of this subsampling
related, high wavenumber bias through an inner/outer loop optimization
framework, the problem persists.
Many neural network emulators use subsampled datasets for training, including
most prominently the ERA5 Reanalysis.
While our work suggests that there could be a benefit to using a training
dataset based on a relatively
shorter model trajectory that is not subsampled, rather than a longer one that
is, addressing the subsampling issue would provide more confidence in using
already existing, freely available datasets like reanalyses.
We therefore suggest that future work should focus on how techniques like
attention or adversarial training can address this subsampling related bias at
the small spatial scales of turbulent geophysical fluid dynamics.
